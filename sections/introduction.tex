\section{Introduction}
\label{sec:introduction}

This exercise is split into two parts. The former half deals with generating pseudo-random number datasets that follow particular distributions. This is implemented analytically for a sinusoidal distribution, and for a general distribution using the reject-accept method. The latter half implements a Monte Carlo simulation of an unstable nucleus decaying and emitting a $\gamma$-ray, taking advantage of the analytic sinusoidal method used in the first half.

The focus of this exercise is in random number generation, in particular with reference to Monte Carlo techniques. Monte Carlo techniques involve the generation of random or pseudo-random numbers over a certain domain with a given Probability Density Function (PDF), the classic example being calculating pi by calculating what ratio of evenly distributed random numbers fall in the area of a circle to a square\cite{Mathews04}. The general random number distributor implemented in the first part of this exercise has the purpose of providing the random number distribution that follows the desired PDF. The latter part applies the Monte Carlo technique as specified here to model the decay of an unstable nucleus travelling towards a detector. The decay time (and therefore position) and $\gamma$-ray decay angles are modelled as randomly generated numbers and the resulting detector hits are calculated trigonometrically.
